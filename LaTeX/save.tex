

\subsection{Abordagens com Algoritmos de Enxame de Partículas Modificados}


No trabalho de \citeonline{psohibrido}, é apresentado um algoritmo híbrido de otimização composto pela metaheurística de PSO Discreta e pelo método de Newton-Raphson (NR). Neste método, o algoritmo de PSO é encarregado de otimizar a função objetivo do problema onde as partículas são vetores de variáveis independentes (magnitude de tensão, potências ativas dos geradores, \emph{taps} e susceptâncias \emph{shunt}). Já o método NR é utilizado para resolver o problema de FP para cada partícula a cada iteração, determinando os valores das variáveis dependentes (ângulos de tensão e potências reativas dos geradores) e satisfazendo as restrições dos balanços de potência ativa e reativa. As variáveis discretas são tratadas através de um operador de arredondamento, de forma que a cada iteração todas as partículas passam por um processo de arredondamento dos valores dos \emph{taps} e susceptâncias \emph{shunt} para o mais próximo possível dentro do conjunto de valores permitidos. Caso alguma partícula viole os limites das restrições de desigualdade do problema, seu valor é restaurado para o valor mais recente factível de sua memória. O método foi aplicado no sistema IEEE de 30 barras no problema de DE, considerando-se a função custo de geração quadrática, custo de geração com PCV, perdas e emissão de gases poluentes oriundos do processo de queima de combustíveis fósseis.


\citeonline{PG-PSO-NR}, desenvolveram um algoritmo modificado da metaheurística PSO e o método de NR para resolução do problema de DE multiobjetivo, visando a minimização do custo de geração quadrático e da EGP. No entanto, para aprimorar a metaheurística de PSO, modificaram o algoritmo original através da inserção do fator de constrição e da técnica de pseudo-gradiente. O fator de constrição é uma modificação no cálculo da velocidade que melhora a capacidade de convergência das partículas do enxame, enquanto que a técnica de pseudo-gradiente é utilizada para determinar a direção mais promissora em problemas de otimização com funções não diferenciáveis, aprimorando o processo de busca no hiperespaço de soluções. O método foi aplicado ao sistema IEEE de 30 barras, considerando-se todas as variáveis do problema como sendo contínuas. As equações do FP do problema são satisfeitas através da execução do método de NR, enquanto que as restrições de desigualdade são tratadas através da penalização na função \emph{fitness}. Além disso, os autores utilizaram um mecanismo baseado na técnica \emph{fuzzy} para determinar a melhor solução de compromisso entre o custo de geração e a emissão de gases poluentes. Embora o algoritmo proposto tenha obtido bons resultados, vale ressaltar que sua aplicação ficou restrita a um problema de pequeno porte onde não são consideradas as variáveis discretas. O problema formulado também não leva em consideração os efeitos de PCV e ZOP.


\citeonline{ga_ip_pt} criaram um algoritmo híbrido da metaheurística GA com o método determinístico contínuo de pontos interiores (MPI) para resolução do FPOR com variáveis discretas e contínuas. Nesta abordagem, o MPI é embutido no interior do GA de forma a ser executado para todos os indivíduos da população, a cada iteração. Enquanto o MPI otimiza apenas as variáveis contínuas e lida com as restrições do problema, o GA é encarregado de otimizar as variáveis discretas (\emph{taps} e susceptâncias \emph{shunt} dos bancos de capacitores e reatores), mantendo-se as contínuas fixadas. O método foi aplicado no sistemas Goiás-Brasília (GO-DF) de 13 barras e IEEE de 118 barras, visando minimizar as perdas de potência ativa nas linhas de transmissão.

No trabalho de \citeonline{EP-SQP}, uma abordagem híbrida foi desenvolvida combinando-se a metaheurística de programação evolutiva (do inglês \emph{evolutionary programming} ou EP) com o método determinístico SQP. O Método híbrido proposto foi aplicado nos casos teste de 13 e 40 geradores sem representação do sistema de transmissão, considerando-se apenas os limites de taxa de rampa e geração. Neste método, no primeiro estágio, o algoritmo EP é utilizado como uma ferramenta de otimização que realiza uma busca global no hiperespaço de soluções. Durante sua execução, as restrições são tratadas através de penalizações na função \emph{fitness}. No segundo estágio, o método de SQP é utilizado para aperfeiçoar localmente a solução encontrada pela metaheurística. Para tratar a não diferenciabilidade da função objetivo, o método de SQP utiliza a técnica denominada diferenças finitas.


No mesmo ano, \citeonline{PE-IP} aplicaram um método híbrido de MPI com EP, para resolução do problema de FPOR com variáveis discretas, visando a minimização das perdas de potência ativa. Na abordagem proposta, inicialmente, o MPI é aplicado ao problema em sua forma relaxada, considerando-se todas as variáveis como sendo contínuas. Em seguida, a solução encontrada pelo MPI é arredondada e utilizada como sendo um dos indivíduos da população inicial do algoritmo EP, que é responsável por realizar uma busca global direcionada a esta solução. O método foi aplicado nos sistemas IEEE de 118 barras e Chongqing (China). As restrições de desigualdade, na metaheurística, são tratadas através da penalização na função \emph{fitness}, enquanto que as equações dos balanços de potência ativa e reativa são satisfeitas por um algoritmo de resolução do FP.


\citeonline{ga_ip}, inspirados no trabalho de \citeonline{ga_ip_pt}, desenvolveram um algoritmo híbrido que combina o MPI com a metaheurística GA para resolução do problema de FPOR com variáveis contínuas e discretas. O método foi aplicado nos sistemas IEEE de 30 e 118 barras, e no sistema Chongqing de 161 barras (China). Inicialmente, o MPI é aplicado ao problema em sua forma relaxada, considerando-se todas as variáveis como sendo contínuas. Em seguida, a solução encontrada pelo MPI é passada como um indivíduo da população do algoritmo GA, no entanto, as variáveis originalmente contínuas do problema são fixadas, de forma que a metaheurística lida apenas com as variáveis discretas. Após a etapa de otimização realizada pelo GA, a solução encontrada é repassada ao MPI, no entanto, agora com as variáveis discretas fixadas e as contínuas livres para serem otimizadas. O processo repete-se até que o critério de parada seja satisfeito. As restrições de desigualdade do problema são tratadas, no GA, através da função \emph{fitness} que considera eventuais violações como sendo penalizações que se somam à função objetivo. As restrições de igualdade são satisfeitas através da utilização de um algoritmo para resolução do FP, executado para cada indivíduo da população.


Semelhante ao proposto por \citeonline{psohibrido}, \citeonline{antlion} programaram um método híbrido utilizando a metaheurística da formiga-leão (do inglês \emph{ant lion optimization} ou ALO) com o método de NR. A metaheurística é utilizada como o elemento responsável por otimizar a função objetivo, enquanto que o método de NR é utilizado para resolver o problema de FP, para cada formiga a cada iteração. As restrições de desigualde são tratadas no algoritmo ALO através da técnica de penalização na função \emph{fitness}, enquanto que as variáveis discretas são tratadas através de um operador de arredondamento. O método foi aplicado nos sistemas IEEE de 30, 118 e 300 barras, nas seguintes funções objetivo: minimização das perdas, minimização do desvio de tensão e melhoria do índice de estabilidade de tensão.
No trabalho de \citeonline{IP-DE} é aplicado um algoritmo híbrido envolvendo o método determinístico contínuo MPI e a metaheurística ED no problema de DE-PCV, sem a representação do sistema de transmissão. Na metodologia proposta, inicialmente é executado o MPI para o problema de DE considerando-se a função objetivo como sendo o custo de geração quadrático, sem os efeitos de PCV. A solução encontrada é então passada para a metaheurística ED, que inicializa sua população ao redor deste ponto utilizando uma técnica específica que leva em consideração o período de variação da função seno presente no custo de geração com efeito de PCV. A abordagem foi aplicada na resolução de três casos distintos utilizando 13 e 40 geradores, obtendo resultados melhores que os de outros métodos híbridos ou metaheurísticos encontrados na literatura. 